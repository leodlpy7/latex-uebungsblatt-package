\documentclass{article}
\usepackage{uebungsblatt}
\usepackage{lipsum}
\begin{document}
    % set all parameters for the exercise sheet
    \setUebungsblattParams[
        name=John Doe, % your name
        matrikelnr=1234567, % your matriculation number
        dozent=Jenny Doe, % your lecturer
        modul=Anatomy, % your lecture
        university=Universidad de los muertos, % your university
        semester=Wintersemester 2024/25, % the semester
        semestershort=WiSe 24/25, % the abbreviation of the semester
        uebungsblattnr=1, % the exercise sheet number
        aufgabennr=0 % the exercise number to start the exercise sheet,
        % because of implementation details, you have to subtract one
    ]{}
    % build the style of the exercise sheet
    \makeUebungsblattStyle{}

    % now begin with the exercises
    % the first parameter is the exercise name, the second the points you can get
    \begin{exercise}{Exercise Name}{10}
        \lipsum[1]
        % you can define subtasks ...
        \begin{subtasks}
            % ... with the subtasks written down in their own environment
            \begin{subtask}{3}
                \lipsum[1]
            \end{subtask}
            \begin{subtask}{3}
                \lipsum[1]
            \end{subtask}
        \end{subtasks}
    \end{exercise}
    % and of course you can give the solution of the exercise
    \begin{solution}
        % with all subtasks like above
        \begin{subtasks}
            \begin{subtask}{}
                \lipsum[1]
            \end{subtask}
            \begin{subtask}{}
                \lipsum[1]
            \end{subtask}
        \end{subtasks}
    \end{solution}
\end{document}